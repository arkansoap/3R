\documentclass{beamer}

\usepackage{tikz}

\usepackage[utf8]{inputenc}
\useinnertheme[shadow=true]{rounded}
\useoutertheme{infolines}
\usecolortheme{beaver}

\setbeamerfont{block title}{size={}}
\setbeamercolor{titlelike}{parent=structure,bg=white}
%title information
\title{Visualisation in econometrics}
\author{FUCHEZ Thibault, \\
CHAVENEAU Lucas, \\ 
GUICHARD Allan}
\date{November 2nd, 2021}

\begin{document}

\maketitle


\begin{frame}{Table of content}
    \tableofcontents[hideallsubsections]{}
\end{frame}

\section{Partie 1 : Covariance}

\subsection{Definition}

\begin{frame}
On commence par définir de manière littéraire la covariance
    \begin{block}{What is Covariance ?}
        A covariance refers to the measure of how two random variables will change when they are compared to each other.
    \end{block}
    \begin{block}{Spearman's rank correlation coefficient}
        Measures the association based on the ranks of the variables.
    \end{block}
    $$
    \hat{\theta}=\frac{\sum_{i=1}^n(R_i-\bar{R}(S_i-\bar{S}))}{\sqrt{\sum_{i=1}^n(R_i-\bar{R})^2\sum_{i=1}^n(S_i-\bar{S})^2}}
    $$

where $R_i$ and $S_i$ are the rank of the $x_i$ and $y_i$ values, respectively.
\end{frame}

\subsection{Hypothesis}

\begin{frame}{A bit Theory}

On écris les hypothèses qui régissent les calculs :\\
\vspace{10}
Assume 2 random variables $X, Y$ and a random bivariate sample $(x_1, y_1),\ (x_2, y_2),$  \ldots , $(x_N, y_N)$. \\
\vspace{10}
Let's consider that the observations are independant of each other, however X, Y could have an impact on each other. 
\end{frame}

\begin{frame}{A bit Theory}
Classical Covariance:
    \begin{equation}
        \sigma^2_x=\frac{1}{n}\sum_{x=1}^{n}(x_i - \bar{x})^2 = \frac{1}{n}\sum_{x=1}^{n}x_i^2 - \bar{x}^2 
    \end{equation}
Hefferman Covariance:
    \begin{equation}
        cov(X,Y)= \frac{2}{n(n-1)}\sum_{i=1}^{n-1}\sum_{j>i}^{n}\frac{1}{2}(x_i-x_j)(y_i - y_j)
    \end{equation}
\end{frame}

\subsection{Issues about the covariance graphical representation}
\begin{frame}{How it looks ?!}
Montrer quelques représentations de la Covariance : \\

The first use of Venn Diagrams : Cohen, J., and Cohen, P. (1975), Applied Multiple Regression/Correlation Analysis for the Behavioral Sciences, Hillside, NJ: Lawrence Erlbaum Associates.
Peter Kennedy : 2001 : Venn Diagrams for Regression
Kevin Hayes : Mai 2011 : Covariance triangle rectangle
Scatter_Plot : Qui l'a fait en premier ?? Article qui relate la covariance rectangle ?

L'étape d'après ? 
\begin{itemize}
    \item Covariance retangle très coûteuse en terme de performance ! 
\end{itemize}





\end{frame}

\subsection{}



\subsection{Visualisation in Econometrics}

\begin{frame}{Contextualisation}
Explain the issues about the graphic representation in econometrics from the start: 

What are the best representation nowadays :


\end{frame}

Lister ici les documents faisant référence à la chronologie de la visualisation en économétrie : 

Diagramme de Venn
Représentation de la covariance en triangle rectangle
Régréssion 
etc... 

\end{document}
